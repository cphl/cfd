% Template adapted from that for Scribe Notes - CS500 (Spring 2016)
% (template is a modified version of one prepared by Erik Demaine at MIT)
%
\documentclass[11pt]{article}
\usepackage{latexsym}
\usepackage{amsmath}
\usepackage{amssymb}
\usepackage{amsthm}
\usepackage{amsfonts}
\usepackage{epsfig}
%\usepackage{psfig}
\usepackage{enumerate}  % use this for lettered points
\usepackage{hyperref}   % to add links

\newcommand{\handout}[5]{
  \noindent
  \begin{center}
  \framebox{
    \vbox{
      \hbox to 5.78in { \textbf{PHYS 447} \hfill #2 }
      \vspace{4mm}
      \hbox to 5.78in { {\Large \hfill #5  \hfill} }
      \vspace{2mm}
      \hbox to 5.78in { \emph{#3 \hfill #4} }
    }
  }
  \end{center}
  \vspace*{4mm}
}

\newcommand{\lecture}[4]{\handout{#1}{#2}{#3}{-: #4}{#1}}
\newcommand{\R}{\mathbb{R}}     % use for capital R real number set symbol
\newcommand{\Z}{\mathbb{Z}}     % use for capital Z integer set symbol

% tabbing environment for pseudo-code 
\newenvironment{code} {
\begin{tt}
\begin{tabbing}
\ \ \ \ \= \ \ \ \ \= \ \ \ \ \= \ \ \ \ \= \ \ \ \ \= \ \ \ \ \=  \ \ \ \ \= \ \ \ \ \= \\ \kill
}{
\end{tabbing}
\end{tt}
}

\newtheorem{theorem}{Theorem}
\newtheorem{corollary}[theorem]{Corollary}
\newtheorem{lemma}[theorem]{Lemma}
\newtheorem{observation}[theorem]{Observation}
\newtheorem{proposition}[theorem]{Proposition}
\newtheorem{definition}[theorem]{Definition}
\newtheorem{claim}[theorem]{Claim}
\newtheorem{fact}[theorem]{Fact}
\newtheorem{assumption}[theorem]{Assumption}

% 1-inch margins, from fullpage.sty by H.Partl, Version 2, Dec. 15, 1988.
\topmargin 0pt
\advance \topmargin by -\headheight
\advance \topmargin by -\headsep
\textheight 8.9in
\oddsidemargin 0pt
\evensidemargin \oddsidemargin
\marginparwidth 0.5in
\textwidth 6.5in

\parindent 0in
\parskip 1.5ex
%\renewcommand{\baselinestretch}{1.25}

\begin{document}

\lecture{Computational fluid dynamics}{Spring 2016}{}{}

% =========================================================================== %

\section{Introduction}

\vspace{1cm}
This intent of this project was to learn about the physics of fluids and
observe the behaviour in computer simulations.
The process was carried out by starting with simplifying assumptions on the
model, choosing a discretization method, then writing code in Python. At each
following step, an assumption would be generalized so that the change could be
observed. The equations were examined in 1-dimension first so an example will
be shown before the transition to 2-dimensions.

Fluid flow equations addressed:
\begin{itemize}\itemsep0em 
\item linear convection
\item nonlinear convection
\item diffusion
\item Burgers equation
\item Laplace, Poisson, Navier-Stokes
\end{itemize}

... Applications...

... Navier-stokes: Channel / cavity flow ...

...

\section{Physics}

\subsection{Introduction to models in 1-dimensional form}
Beginning with 1-dimensional models, we can describe the differences
between the behaviours of the systems dictated by these equations. They
have their analogues in 2-dimensions.

Starting off, the simplest model looked at was the 1-dimensional equation
describing convective flow without diffusion. It describes the propagation
of a wave with constant speed $c$. There is no deformation of shape of the
initial condition's wave as it propagates. \ref{1d_linconv}

In the nonlinear case, the constant speed is replaced by $u$ which is the
solution $u(x,t)=u_0(x-ct)$ given an initial condition $u(x,0)=u_0(x)$
The speed of propagation is different depending upon the location.
Because of this, shocks may be observed (see corresponding treatment in the
numerical considerations section \ref{}). It is also known as the inviscid
Burgers equation.
\ref{1d_nonlinconv}

Then diffusion is considered without any convection.
If $u$ is temperature, then this is the heat equation.
It is governed by a second-order partial derivative. \ref{1d_diffusion}

For the last of the equations with time evolution, nonlinear convection
and diffusion are treated simultaneously. Since viscosity is now accounted
for with the diffusion term, this is the general Burgers equation. It has
applications outside of fluid dynamics, for example in acoustics.
\ref{1d_Burgers}

\begin{align}
& \text{linear convection} &
\frac{\partial u}{\partial t} + c \frac{\partial u}{\partial x} &= 0		\label{1d_linconv}
\\ \nonumber \\ & \text{(inviscid Burgers) nonlinear convection} &
\frac{\partial u}{\partial t} + u \frac{\partial u}{\partial x} &= 0		\label{1d_nonlinconv}
\\ \nonumber \\ & \text{linear diffusion} &
\frac{\partial u}{\partial t} &= \nu \frac{\partial^2 u}{\partial x^2}		\label{1d_diffusion}
\\ \nonumber \\ & \text{Burgers' equation} &
\frac{\partial u}{\partial t} + u \frac{\partial u}{\partial x} &= \nu \frac{\partial^2 u}{\partial x^2}	\label{1d_Burgers}
\end{align}

% this is useful for spacing equations when they're too far apart... or something
\hspace{1cm}
% equation stuff
%\begin{align*}
%\end{align*}

\subsection{Extension to 2-dimensional models}
The expressions introduced in their 1-dimensional form are now shown
extended to two dimensions.
\subsubsection{Linear convection (2d)}
The conversion from one- to two-dimensions is straightforward for linear
convection. As two spatial components are changing, it is sufficient to
duplicate the original spatial variation in the new dimension.
\begin{align}
\frac{\partial u}{\partial t} + c\left(\frac{\partial u}{\partial x}  + \frac{\partial u}{\partial x} \right)
&= 0		\label{2d_linconv}
\end{align}

\subsubsection{Nonlinear convection (2d)}
Nonlinear in an extra dimension requires a different treatment.

...[more steps here]
\begin{align}
\frac{\partial u}{\partial t} + u\frac{\partial u}{\partial x}  + v\frac{\partial u}{\partial x} 
&= 0		\nonumber \\ \nonumber\\
\frac{\partial v}{\partial t} + u\frac{\partial v}{\partial x}  + v\frac{\partial v}{\partial x} 
&= 0
	\label{2d_nonlinconv}
\end{align}

\subsubsection{Diffusion (2d)}
Diffusion affects flow equally in both components in two dimensions, so
again, a duplication in the y-component works.
\begin{align}
\frac{\partial u}{\partial t} &= \nu \left(\frac{\partial^2 u}{\partial x^2} + \frac{\partial^2 u}{\partial y^2}\right)		\label{2d_diffusion}
\end{align}

\subsubsection{Burgers (2d)}
As in the case with nonlinear convection, coupled PDEs are required to
describe the system.

...
[ more here]

\begin{align}
\frac{\partial u}{\partial t} + u \frac{\partial u}{\partial x} + v \frac{\partial u}{\partial y}
&= \nu \left(\frac{\partial^2 u}{\partial x^2} + \frac{\partial^2 u}{\partial y^2}\right)
\nonumber \\ \nonumber \\
\frac{\partial v}{\partial t} + u \frac{\partial v}{\partial x} + v \frac{\partial v}{\partial y}
&= \nu \left(\frac{\partial^2 v}{\partial x^2} + \frac{\partial^2 v}{\partial y^2}\right)	\label{2d_Burgers}
\end{align}

\subsubsection{Laplace}
The Laplace equation shows a steady state of a system given boundary
conditions. There is no time parameter.
\begin{align}
\frac{\partial^2 p}{\partial x^2} + \frac{\partial^2 p}{\partial y} &= 0
\label{laplace}
\end{align}

\subsubsection{Poisson}
The Poission equation is relevant as the pressure term that will be used in
the Navier-Stokes equation. Compare to the Laplace equation.\ref{laplace}...
The source term $b$ causes the equation to relax initial sources in the field.
... [more on relaxation]...

\begin{align}
\frac{\partial^2 p}{\partial x^2} + \frac{\partial^2 p}{\partial y} &= b
\end{align}

\subsubsection{Channel flow}
Pressure-driven

...

Added source term

...


\section{Numerical considerations}

\subsection{Preparing discretized computation}

Describe general strategy used for all

Do whole process for each one? Or do one and show only results for rest?

\begin{enumerate}[(A)]
\item Discretize

[..]

\item Transpose

[..]

\subsection{CFL condition}
choosing constant for CFL, motivation

\end{enumerate}

\section{Coding strategy}

storage of solutions at each time step

sampling of solutions

single plots vs animations

\subsection{Optimizations}

modularizing with parametrized functions

vectorization to remove nested loop



\subsection{Results}

Show example figures:

linear convection: note false diffusion caused by roundoff in computations

nonlinear

diffusion alone

Burgers with and without shocks

show artefacts from choosing bad timesteps

channel flow

%	\begin{figure}[qfc]
%	\centering
%	\caption{Quantum Factoring Circuit}
%	\includegraphics[scale=0.8]{QuantumFactoringCircuit.eps}
%	\label{fig:qfg}
%	\end{figure}

\section*{References}

\end{document}
